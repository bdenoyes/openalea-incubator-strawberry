\documentclass[]{article}
\usepackage{lmodern}
\usepackage{amssymb,amsmath}
\usepackage{ifxetex,ifluatex}
\usepackage{fixltx2e} % provides \textsubscript
\ifnum 0\ifxetex 1\fi\ifluatex 1\fi=0 % if pdftex
  \usepackage[T1]{fontenc}
  \usepackage[utf8]{inputenc}
\else % if luatex or xelatex
  \ifxetex
    \usepackage{mathspec}
  \else
    \usepackage{fontspec}
  \fi
  \defaultfontfeatures{Ligatures=TeX,Scale=MatchLowercase}
\fi
% use upquote if available, for straight quotes in verbatim environments
\IfFileExists{upquote.sty}{\usepackage{upquote}}{}
% use microtype if available
\IfFileExists{microtype.sty}{%
\usepackage{microtype}
\UseMicrotypeSet[protrusion]{basicmath} % disable protrusion for tt fonts
}{}
\usepackage[margin= 2cm]{geometry}
\usepackage{hyperref}
\hypersetup{unicode=true,
            pdftitle={Architecture\_analysis},
            pdfauthor={Marc LABADIE},
            pdfborder={0 0 0},
            breaklinks=true}
\urlstyle{same}  % don't use monospace font for urls
\usepackage{color}
\usepackage{fancyvrb}
\newcommand{\VerbBar}{|}
\newcommand{\VERB}{\Verb[commandchars=\\\{\}]}
\DefineVerbatimEnvironment{Highlighting}{Verbatim}{commandchars=\\\{\}}
% Add ',fontsize=\small' for more characters per line
\usepackage{framed}
\definecolor{shadecolor}{RGB}{248,248,248}
\newenvironment{Shaded}{\begin{snugshade}}{\end{snugshade}}
\newcommand{\KeywordTok}[1]{\textcolor[rgb]{0.13,0.29,0.53}{\textbf{#1}}}
\newcommand{\DataTypeTok}[1]{\textcolor[rgb]{0.13,0.29,0.53}{#1}}
\newcommand{\DecValTok}[1]{\textcolor[rgb]{0.00,0.00,0.81}{#1}}
\newcommand{\BaseNTok}[1]{\textcolor[rgb]{0.00,0.00,0.81}{#1}}
\newcommand{\FloatTok}[1]{\textcolor[rgb]{0.00,0.00,0.81}{#1}}
\newcommand{\ConstantTok}[1]{\textcolor[rgb]{0.00,0.00,0.00}{#1}}
\newcommand{\CharTok}[1]{\textcolor[rgb]{0.31,0.60,0.02}{#1}}
\newcommand{\SpecialCharTok}[1]{\textcolor[rgb]{0.00,0.00,0.00}{#1}}
\newcommand{\StringTok}[1]{\textcolor[rgb]{0.31,0.60,0.02}{#1}}
\newcommand{\VerbatimStringTok}[1]{\textcolor[rgb]{0.31,0.60,0.02}{#1}}
\newcommand{\SpecialStringTok}[1]{\textcolor[rgb]{0.31,0.60,0.02}{#1}}
\newcommand{\ImportTok}[1]{#1}
\newcommand{\CommentTok}[1]{\textcolor[rgb]{0.56,0.35,0.01}{\textit{#1}}}
\newcommand{\DocumentationTok}[1]{\textcolor[rgb]{0.56,0.35,0.01}{\textbf{\textit{#1}}}}
\newcommand{\AnnotationTok}[1]{\textcolor[rgb]{0.56,0.35,0.01}{\textbf{\textit{#1}}}}
\newcommand{\CommentVarTok}[1]{\textcolor[rgb]{0.56,0.35,0.01}{\textbf{\textit{#1}}}}
\newcommand{\OtherTok}[1]{\textcolor[rgb]{0.56,0.35,0.01}{#1}}
\newcommand{\FunctionTok}[1]{\textcolor[rgb]{0.00,0.00,0.00}{#1}}
\newcommand{\VariableTok}[1]{\textcolor[rgb]{0.00,0.00,0.00}{#1}}
\newcommand{\ControlFlowTok}[1]{\textcolor[rgb]{0.13,0.29,0.53}{\textbf{#1}}}
\newcommand{\OperatorTok}[1]{\textcolor[rgb]{0.81,0.36,0.00}{\textbf{#1}}}
\newcommand{\BuiltInTok}[1]{#1}
\newcommand{\ExtensionTok}[1]{#1}
\newcommand{\PreprocessorTok}[1]{\textcolor[rgb]{0.56,0.35,0.01}{\textit{#1}}}
\newcommand{\AttributeTok}[1]{\textcolor[rgb]{0.77,0.63,0.00}{#1}}
\newcommand{\RegionMarkerTok}[1]{#1}
\newcommand{\InformationTok}[1]{\textcolor[rgb]{0.56,0.35,0.01}{\textbf{\textit{#1}}}}
\newcommand{\WarningTok}[1]{\textcolor[rgb]{0.56,0.35,0.01}{\textbf{\textit{#1}}}}
\newcommand{\AlertTok}[1]{\textcolor[rgb]{0.94,0.16,0.16}{#1}}
\newcommand{\ErrorTok}[1]{\textcolor[rgb]{0.64,0.00,0.00}{\textbf{#1}}}
\newcommand{\NormalTok}[1]{#1}
\usepackage{longtable,booktabs}
\usepackage{graphicx,grffile}
\makeatletter
\def\maxwidth{\ifdim\Gin@nat@width>\linewidth\linewidth\else\Gin@nat@width\fi}
\def\maxheight{\ifdim\Gin@nat@height>\textheight\textheight\else\Gin@nat@height\fi}
\makeatother
% Scale images if necessary, so that they will not overflow the page
% margins by default, and it is still possible to overwrite the defaults
% using explicit options in \includegraphics[width, height, ...]{}
\setkeys{Gin}{width=\maxwidth,height=\maxheight,keepaspectratio}
\IfFileExists{parskip.sty}{%
\usepackage{parskip}
}{% else
\setlength{\parindent}{0pt}
\setlength{\parskip}{6pt plus 2pt minus 1pt}
}
\setlength{\emergencystretch}{3em}  % prevent overfull lines
\providecommand{\tightlist}{%
  \setlength{\itemsep}{0pt}\setlength{\parskip}{0pt}}
\setcounter{secnumdepth}{5}
% Redefines (sub)paragraphs to behave more like sections
\ifx\paragraph\undefined\else
\let\oldparagraph\paragraph
\renewcommand{\paragraph}[1]{\oldparagraph{#1}\mbox{}}
\fi
\ifx\subparagraph\undefined\else
\let\oldsubparagraph\subparagraph
\renewcommand{\subparagraph}[1]{\oldsubparagraph{#1}\mbox{}}
\fi

%%% Use protect on footnotes to avoid problems with footnotes in titles
\let\rmarkdownfootnote\footnote%
\def\footnote{\protect\rmarkdownfootnote}

%%% Change title format to be more compact
\usepackage{titling}

% Create subtitle command for use in maketitle
\newcommand{\subtitle}[1]{
  \posttitle{
    \begin{center}\large#1\end{center}
    }
}

\setlength{\droptitle}{-2em}
  \title{Architecture\_analysis}
  \pretitle{\vspace{\droptitle}\centering\huge}
  \posttitle{\par}
  \author{Marc LABADIE}
  \preauthor{\centering\large\emph}
  \postauthor{\par}
  \predate{\centering\large\emph}
  \postdate{\par}
  \date{5 juin 2018}


\begin{document}
\maketitle

{
\setcounter{tocdepth}{4}
\tableofcontents
}
\section{Requierements}\label{requierements}

\subsection{Packages install}\label{packages-install}

\begin{Shaded}
\begin{Highlighting}[]
\KeywordTok{install.packages}\NormalTok{(}\StringTok{"plyr"}\NormalTok{)}
\KeywordTok{install.packages}\NormalTok{(}\StringTok{"ggplot2"}\NormalTok{)}
\KeywordTok{install.packages}\NormalTok{(}\StringTok{"gtable"}\NormalTok{)}
\KeywordTok{install.packages}\NormalTok{(}\StringTok{"grid"}\NormalTok{)}
\KeywordTok{install.packages}\NormalTok{(}\StringTok{"cowplot"}\NormalTok{)}
\KeywordTok{install.packages}\NormalTok{(}\StringTok{"reshape2"}\NormalTok{)}
\KeywordTok{install.packages}\NormalTok{(}\StringTok{"scales"}\NormalTok{)}
\KeywordTok{install.packages}\NormalTok{(}\StringTok{"knitr"}\NormalTok{)}
\KeywordTok{install.packages}\NormalTok{(}\StringTok{"tinytex"}\NormalTok{)}
\KeywordTok{install.packages}\NormalTok{(}\StringTok{"dplyr"}\NormalTok{)}
\CommentTok{#install.packages("RCurl")}
\end{Highlighting}
\end{Shaded}

\subsection{Packages loading}\label{packages-loading}

\begin{Shaded}
\begin{Highlighting}[]
\KeywordTok{library}\NormalTok{(plyr)}
\KeywordTok{library}\NormalTok{(ggplot2)}
\KeywordTok{library}\NormalTok{(gtable)}
\KeywordTok{library}\NormalTok{(grid)}
\KeywordTok{library}\NormalTok{(cowplot)}
\KeywordTok{library}\NormalTok{(reshape2)}
\KeywordTok{library}\NormalTok{(scales)}
\KeywordTok{library}\NormalTok{(knitr)}
\KeywordTok{library}\NormalTok{(tinytex)}
\KeywordTok{library}\NormalTok{(dplyr)}
\CommentTok{#library(RCurl)}
\KeywordTok{library}\NormalTok{(rmarkdown)}
\end{Highlighting}
\end{Shaded}

\subsection{Functions importing}\label{functions-importing}

\begin{Shaded}
\begin{Highlighting}[]
\KeywordTok{source}\NormalTok{(}\DataTypeTok{file =} \StringTok{"c:/Users/mlabadie/Documents/GitHub/strawberry/Rscript/Functions.R"}\NormalTok{)}
\end{Highlighting}
\end{Shaded}

\section{Import and transformation of
dataset}\label{import-and-transformation-of-dataset}

\subsection{Import dataset}\label{import-dataset}

INDEX\_PARAMETER : TIME \# vertex\_id

15 VARIABLES

VARIABLE 1 : INT \# nb\_visible\_leaves. No. elongated leaves (F)
VARIABLE 2 : INT \# nb\_foliar\_primordia No. primordia (f) VARIABLE 3 :
INT \# nb\_total\_leaves. Total no. leaves (F+f) VARIABLE 4 : INT \#
nb\_open\_flowers. No. open flowers VARIABLE 5 : INT \#
nb\_aborted\_flowers No. aborted flowers VARIABLE 6 : INT \#
nb\_total\_flowers Total no. flowers VARIABLE 7 : INT \#
vegetative\_bud. No. vegetative buds (axillary vegetative bud) VARIABLE
8 : INT \# Initiated\_bud. No. initiated bud (axillary initiated bud)
VARIABLE 9 : INT \# floral\_bud. No. floral buds (axillary floral bud)
VARIABLE 10 : INT \# stolons No. stolons VARIABLE 11 : INT \#
type\_of\_crown. Type of crown (1: primary crown, 2: extention crowns,
3: branch crown) VARIABLE 12 : INT \# Crown\_status (1: Terminal
Vegetative bud (bt, stage 17, 18, 19, None), 2:Terminal bud initiated
(bt, stage A), 3: Terminal floral bud(ht), 4: Inflorescence(HT), -1:
rotten or aborded) VARIABLE 13 : INT \# genotype (1: Gariguette, 2:
Ciflorette, 3: Clery, 4: Capriss, 5:Darselect, 6: Cir107) VARIABLE 14 :
INT \# date (1: mid December, 2: early Junuary, 3: mid February, 4:
early March, 5: early April, 6: end May/early June) VARIABLE 15 : INT \#
plant. plant index

\begin{Shaded}
\begin{Highlighting}[]
\NormalTok{DataSet <-}\StringTok{ }\KeywordTok{read.csv}\NormalTok{(}
  \StringTok{"c:/Users/mlabadie/Documents/GitHub/strawberry/Rscript/Dataset.csv"}\NormalTok{, }
  \DataTypeTok{sep=}\StringTok{";"}\NormalTok{,}\DataTypeTok{na.strings =} \StringTok{"-1"}\NormalTok{)}

\NormalTok{colstart<-}\DecValTok{1}
\NormalTok{colend<-}\KeywordTok{dim}\NormalTok{(DataSet)[}\DecValTok{2}\NormalTok{]}\OperatorTok{-}\DecValTok{2}

\NormalTok{data<-DataSet[,}\KeywordTok{c}\NormalTok{(colstart}\OperatorTok{:}\NormalTok{colend)]}
\end{Highlighting}
\end{Shaded}

\subsection{Dataset Class Object}\label{dataset-class-object}

\begin{Shaded}
\begin{Highlighting}[]
\KeywordTok{str}\NormalTok{(}\DataTypeTok{object =}\NormalTok{ data)}
\end{Highlighting}
\end{Shaded}

\begin{verbatim}
## 'data.frame':    1796 obs. of  16 variables:
##  $ Index              : Factor w/ 17 levels "0","0-1","0-1-2",..: 1 1 1 1 1 1 1 1 1 1 ...
##  $ nb_visible_leaves  : int  8 8 11 8 6 7 7 6 7 11 ...
##  $ nb_foliar_primordia: int  4 4 3 3 4 4 4 3 3 8 ...
##  $ nb_total_leaves    : int  12 12 14 11 10 11 11 9 10 19 ...
##  $ nb_open_flowers    : int  0 0 0 0 0 0 0 0 0 0 ...
##  $ nb_aborted_flowers : int  0 0 0 0 0 0 0 0 0 0 ...
##  $ nb_total_flowers   : int  0 0 0 0 0 0 0 0 0 0 ...
##  $ vegetative_bud     : int  1 4 3 6 5 3 7 1 4 3 ...
##  $ Initiated_bud      : int  3 3 1 0 2 2 1 3 0 2 ...
##  $ floral_bud         : int  7 4 8 5 2 5 2 4 5 10 ...
##  $ stolons            : int  1 1 2 0 1 1 1 1 1 1 ...
##  $ type_of_crown      : int  1 1 1 1 1 1 1 1 1 1 ...
##  $ Crown_status       : int  3 3 3 3 3 3 3 3 3 3 ...
##  $ genotype           : int  1 1 1 1 1 1 1 1 1 1 ...
##  $ date               : int  1 1 1 1 1 1 1 1 1 2 ...
##  $ plant              : int  1 2 3 4 5 6 7 8 9 1 ...
\end{verbatim}

\subsection{Transformation of Class
object}\label{transformation-of-class-object}

\begin{Shaded}
\begin{Highlighting}[]
\NormalTok{data}\OperatorTok{$}\NormalTok{genotype<-}\StringTok{ }\KeywordTok{as.factor}\NormalTok{(data}\OperatorTok{$}\NormalTok{genotype)}
\NormalTok{data}\OperatorTok{$}\NormalTok{date<-}\StringTok{ }\KeywordTok{as.factor}\NormalTok{(DataSet}\OperatorTok{$}\NormalTok{date)}
\NormalTok{data}\OperatorTok{$}\NormalTok{Crown_status<-}\StringTok{ }\KeywordTok{as.factor}\NormalTok{(DataSet}\OperatorTok{$}\NormalTok{Crown_status)}
\NormalTok{data}\OperatorTok{$}\NormalTok{type_of_crown<-}\StringTok{ }\KeywordTok{as.factor}\NormalTok{(DataSet}\OperatorTok{$}\NormalTok{type_of_crown)}
\end{Highlighting}
\end{Shaded}

\subsection{Conversion of dataset}\label{conversion-of-dataset}

\begin{Shaded}
\begin{Highlighting}[]
\CommentTok{# Convert numerical categorical ordered value in factor values with their properties }

\NormalTok{data}\OperatorTok{$}\NormalTok{genotype<-}\StringTok{ }\KeywordTok{factor}\NormalTok{(}\DataTypeTok{x =}\NormalTok{ data}\OperatorTok{$}\NormalTok{genotype,}
                       \DataTypeTok{levels =} \KeywordTok{levels}\NormalTok{(}\DataTypeTok{x =}\NormalTok{ data}\OperatorTok{$}\NormalTok{genotype),}
                       \DataTypeTok{labels =} \KeywordTok{c}\NormalTok{(}\StringTok{"Gariguette"}\NormalTok{,}\StringTok{"Ciflorette"}\NormalTok{,}\StringTok{"Clery"}\NormalTok{,}\StringTok{"Capriss"}\NormalTok{,}\StringTok{"Darselect"}\NormalTok{,}\StringTok{"Cir107"}\NormalTok{)}
\NormalTok{                       )}

\NormalTok{data}\OperatorTok{$}\NormalTok{date<-}\StringTok{ }\KeywordTok{factor}\NormalTok{(}\DataTypeTok{x =}\NormalTok{ data}\OperatorTok{$}\NormalTok{date,}
                   \DataTypeTok{levels =} \KeywordTok{levels}\NormalTok{(}\DataTypeTok{x =}\NormalTok{ data}\OperatorTok{$}\NormalTok{date),}
                   \DataTypeTok{labels =} \KeywordTok{c}\NormalTok{(}\StringTok{"Mid-December"}\NormalTok{,}\StringTok{"Early-Junuary"}\NormalTok{,}\StringTok{"Mid-February"}\NormalTok{,}\StringTok{"Early-March"}\NormalTok{,}\StringTok{"Early-April"}\NormalTok{,}\StringTok{"Early-June"}\NormalTok{)}
\NormalTok{                   )}

\NormalTok{data}\OperatorTok{$}\NormalTok{type_of_crown<-}\StringTok{ }\KeywordTok{factor}\NormalTok{(}\DataTypeTok{x =}\NormalTok{ DataSet}\OperatorTok{$}\NormalTok{type_of_crown,}
                            \DataTypeTok{levels =} \KeywordTok{levels}\NormalTok{(}\DataTypeTok{x =}\NormalTok{ data}\OperatorTok{$}\NormalTok{type_of_crown),}
                            \DataTypeTok{labels =} \KeywordTok{c}\NormalTok{(}\StringTok{"Primary_Crown"}\NormalTok{,}\StringTok{"Extention_Crown"}\NormalTok{,}\StringTok{"Branch_Crown"}\NormalTok{)}
\NormalTok{                            )}
\NormalTok{data}\OperatorTok{$}\NormalTok{Crown_status<-}\StringTok{ }\KeywordTok{factor}\NormalTok{(}\DataTypeTok{x =}\NormalTok{ data}\OperatorTok{$}\NormalTok{Crown_status,}
                              \DataTypeTok{levels =} \KeywordTok{levels}\NormalTok{(}\DataTypeTok{x =}\NormalTok{ data}\OperatorTok{$}\NormalTok{Crown_status),}
                              \DataTypeTok{labels =} \KeywordTok{c}\NormalTok{(}\StringTok{"Terminal_Vegetative_bud"}\NormalTok{,}\StringTok{"Terminal_initiated_bud"}\NormalTok{,}\StringTok{"Terminal_Floral_bud"}\NormalTok{,}\StringTok{"Terminal_Inflorescence"}\NormalTok{))}
\end{Highlighting}
\end{Shaded}

\begin{Shaded}
\begin{Highlighting}[]
\CommentTok{#convert index sequence analysis in index for R analysis}
\NormalTok{dat<-data[}\DecValTok{2}\OperatorTok{:}\NormalTok{colend]}
\ControlFlowTok{for}\NormalTok{ (i }\ControlFlowTok{in} \DecValTok{1}\OperatorTok{:}\KeywordTok{nrow}\NormalTok{(data))\{ }
  \ControlFlowTok{if}\NormalTok{ (data[i,}\StringTok{'Index'}\NormalTok{]}\OperatorTok{==}\StringTok{"0"}\NormalTok{)\{ }
\NormalTok{    dat[i,}\StringTok{"Index"}\NormalTok{]<-}\StringTok{ }\DecValTok{0}
\NormalTok{  \}}\ControlFlowTok{else} \ControlFlowTok{if}\NormalTok{ (data[i,}\StringTok{'Index'}\NormalTok{]}\OperatorTok{==}\StringTok{"0-1"}\NormalTok{)\{}
\NormalTok{    dat[i,}\StringTok{"Index"}\NormalTok{]<-}\StringTok{ }\DecValTok{1}
\NormalTok{  \}}\ControlFlowTok{else} \ControlFlowTok{if}\NormalTok{ (data[i,}\StringTok{'Index'}\NormalTok{]}\OperatorTok{==}\StringTok{"0-1-2"}\NormalTok{)\{}
\NormalTok{    dat[i,}\StringTok{"Index"}\NormalTok{]<-}\StringTok{ }\DecValTok{2}
\NormalTok{  \}}\ControlFlowTok{else} \ControlFlowTok{if}\NormalTok{ (data[i,}\StringTok{'Index'}\NormalTok{]}\OperatorTok{==}\StringTok{"0-1-2-3"}\NormalTok{)\{}
\NormalTok{    dat[i,}\StringTok{"Index"}\NormalTok{]<-}\StringTok{ }\DecValTok{3}
\NormalTok{  \}}\ControlFlowTok{else} \ControlFlowTok{if}\NormalTok{ (data[i,}\StringTok{'Index'}\NormalTok{]}\OperatorTok{==}\StringTok{"0-1-2-3-4"}\NormalTok{)\{}
\NormalTok{    dat[i,}\StringTok{"Index"}\NormalTok{]<-}\StringTok{ }\DecValTok{4}
\NormalTok{  \}}\ControlFlowTok{else} \ControlFlowTok{if}\NormalTok{ (data[i,}\StringTok{'Index'}\NormalTok{]}\OperatorTok{==}\StringTok{"0-1-2-3-4-5"}\NormalTok{)\{}
\NormalTok{    dat[i,}\StringTok{"Index"}\NormalTok{]<-}\StringTok{ }\DecValTok{5}
\NormalTok{  \}}\ControlFlowTok{else} \ControlFlowTok{if}\NormalTok{ (data[i,}\StringTok{'Index'}\NormalTok{]}\OperatorTok{==}\StringTok{"1"}\NormalTok{)\{}
\NormalTok{    dat[i,}\StringTok{"Index"}\NormalTok{]<-}\StringTok{ }\DecValTok{1}
\NormalTok{  \}}\ControlFlowTok{else} \ControlFlowTok{if}\NormalTok{ (data[i,}\StringTok{'Index'}\NormalTok{]}\OperatorTok{==}\StringTok{"1-2"}\NormalTok{)\{}
\NormalTok{    dat[i,}\StringTok{"Index"}\NormalTok{]<-}\StringTok{ }\DecValTok{2}
\NormalTok{  \}}\ControlFlowTok{else} \ControlFlowTok{if}\NormalTok{ (data[i,}\StringTok{'Index'}\NormalTok{]}\OperatorTok{==}\StringTok{"1-2-3"}\NormalTok{)\{}
\NormalTok{    dat[i,}\StringTok{"Index"}\NormalTok{]<-}\StringTok{ }\DecValTok{3}
\NormalTok{  \}}\ControlFlowTok{else} \ControlFlowTok{if}\NormalTok{ (data[i,}\StringTok{'Index'}\NormalTok{]}\OperatorTok{==}\StringTok{"1-2-3-4"}\NormalTok{)\{}
\NormalTok{    dat[i,}\StringTok{"Index"}\NormalTok{]<-}\StringTok{ }\DecValTok{4}
\NormalTok{  \}}\ControlFlowTok{else} \ControlFlowTok{if}\NormalTok{ (data[i,}\StringTok{'Index'}\NormalTok{]}\OperatorTok{==}\StringTok{"1-2-3-4-5"}\NormalTok{)\{}
\NormalTok{    dat[i,}\StringTok{"Index"}\NormalTok{]<-}\StringTok{ }\DecValTok{5}
\NormalTok{  \}}\ControlFlowTok{else} \ControlFlowTok{if}\NormalTok{ (data[i,}\StringTok{'Index'}\NormalTok{]}\OperatorTok{==}\StringTok{"2"}\NormalTok{)\{}
\NormalTok{    dat[i,}\StringTok{"Index"}\NormalTok{]<-}\StringTok{ }\DecValTok{2}
\NormalTok{  \}}\ControlFlowTok{else} \ControlFlowTok{if}\NormalTok{ (data[i,}\StringTok{'Index'}\NormalTok{]}\OperatorTok{==}\StringTok{"2-3"}\NormalTok{)\{}
\NormalTok{    dat[i,}\StringTok{"Index"}\NormalTok{]<-}\StringTok{ }\DecValTok{3}
\NormalTok{  \}}\ControlFlowTok{else} \ControlFlowTok{if}\NormalTok{ (data[i,}\StringTok{'Index'}\NormalTok{]}\OperatorTok{==}\StringTok{"2-3-4"}\NormalTok{)\{}
\NormalTok{    dat[i,}\StringTok{"Index"}\NormalTok{]<-}\StringTok{ }\DecValTok{4}
\NormalTok{  \}}\ControlFlowTok{else} \ControlFlowTok{if}\NormalTok{ (data[i,}\StringTok{'Index'}\NormalTok{]}\OperatorTok{==}\StringTok{"3"}\NormalTok{)\{}
\NormalTok{    dat[i,}\StringTok{"Index"}\NormalTok{]<-}\StringTok{ }\DecValTok{3}
\NormalTok{  \}}\ControlFlowTok{else} \ControlFlowTok{if}\NormalTok{ (data[i,}\StringTok{'Index'}\NormalTok{]}\OperatorTok{==}\StringTok{"3-4"}\NormalTok{)\{}
\NormalTok{    dat[i,}\StringTok{"Index"}\NormalTok{]<-}\StringTok{ }\DecValTok{4}
\NormalTok{  \}}\ControlFlowTok{else} \ControlFlowTok{if}\NormalTok{ (data[i,}\StringTok{'Index'}\NormalTok{]}\OperatorTok{==}\StringTok{"4"}\NormalTok{)\{}
\NormalTok{    dat[i,}\StringTok{"Index"}\NormalTok{]<-}\StringTok{ }\DecValTok{4}
\NormalTok{  \}}
\NormalTok{\}}

\NormalTok{dat}\OperatorTok{$}\NormalTok{Index<-}\KeywordTok{as.factor}\NormalTok{(}\DataTypeTok{x =}\NormalTok{ dat}\OperatorTok{$}\NormalTok{Index)}
\end{Highlighting}
\end{Shaded}

\section{Exploratory analysis}\label{exploratory-analysis}

\subsection{At module scale}\label{at-module-scale}

** * Extraction of data at module scale **

\begin{Shaded}
\begin{Highlighting}[]
\NormalTok{data_at_module_scale<-}\KeywordTok{ddply}\NormalTok{(}\DataTypeTok{.data =}\NormalTok{ dat,}
                            \DataTypeTok{.variables =} \KeywordTok{c}\NormalTok{(}\StringTok{"genotype"}\NormalTok{,}\StringTok{"Index"}\NormalTok{),}
\NormalTok{                            summarise,}
                            \DataTypeTok{MeanTotalLeave=} \KeywordTok{round}\NormalTok{(}\DataTypeTok{x =} \KeywordTok{mean}\NormalTok{(}\DataTypeTok{x =}\NormalTok{ nb_total_leaves,}
                                                           \DataTypeTok{na.rm =}\NormalTok{ T),}
                                                  \DataTypeTok{digits =} \DecValTok{0}\NormalTok{),}
                            \DataTypeTok{SdTotalLeave=} \KeywordTok{sd}\NormalTok{(}\DataTypeTok{x =}\NormalTok{ nb_total_leaves,}
                                  \DataTypeTok{na.rm =}\NormalTok{ T),}
                            \DataTypeTok{MeanTotalFlower=} \KeywordTok{round}\NormalTok{(}\KeywordTok{mean}\NormalTok{(}\DataTypeTok{x =}\NormalTok{ nb_total_flowers,}
                                                        \DataTypeTok{na.rm =}\NormalTok{ T),}
                                                   \DataTypeTok{digits =} \DecValTok{0}\NormalTok{),}
                            \DataTypeTok{SdTotalFlower=} \KeywordTok{sd}\NormalTok{(}\DataTypeTok{x =}\NormalTok{ nb_total_flowers,}
                                              \DataTypeTok{na.rm =}\NormalTok{ T),}
                            \DataTypeTok{MeanStolon=} \KeywordTok{round}\NormalTok{(}\KeywordTok{mean}\NormalTok{(}\DataTypeTok{x =}\NormalTok{ stolons,}
                                                   \DataTypeTok{na.rm =}\NormalTok{ T),}
                                              \DataTypeTok{digits =} \DecValTok{0}\NormalTok{),}
                            \DataTypeTok{SdStolon=} \KeywordTok{sd}\NormalTok{(}\DataTypeTok{x =}\NormalTok{ stolons,}
                                         \DataTypeTok{na.rm =}\NormalTok{ T),}
                            \DataTypeTok{N=}\KeywordTok{length}\NormalTok{(nb_total_leaves))}

\KeywordTok{kable}\NormalTok{(}\DataTypeTok{x =}\NormalTok{ data_at_module_scale,}\DataTypeTok{caption =} \StringTok{" Data at module scale"}\NormalTok{)}
\end{Highlighting}
\end{Shaded}

\begin{longtable}[]{@{}llrrrrrrr@{}}
\caption{Data at module scale}\tabularnewline
\toprule
genotype & Index & MeanTotalLeave & SdTotalLeave & MeanTotalFlower &
SdTotalFlower & MeanStolon & SdStolon & N\tabularnewline
\midrule
\endfirsthead
\toprule
genotype & Index & MeanTotalLeave & SdTotalLeave & MeanTotalFlower &
SdTotalFlower & MeanStolon & SdStolon & N\tabularnewline
\midrule
\endhead
Gariguette & 0 & 10 & 2.3520513 & 14 & 8.690441 & 1 & 0.7523548 &
54\tabularnewline
Gariguette & 1 & 3 & 1.2818355 & 7 & 4.365022 & 0 & 0.0000000 &
94\tabularnewline
Gariguette & 2 & 4 & 1.5464772 & 4 & 3.869534 & 0 & 0.6455314 &
62\tabularnewline
Gariguette & 3 & 3 & 0.5117663 & 4 & 3.269629 & 0 & 0.5606119 &
21\tabularnewline
Gariguette & 4 & 3 & 0.4409586 & 5 & 1.166667 & 1 & 0.7817360 &
9\tabularnewline
Gariguette & 5 & 3 & NA & 7 & NA & 2 & NA & 1\tabularnewline
Ciflorette & 0 & 8 & 2.5719554 & 8 & 5.392751 & 1 & 0.9569708 &
54\tabularnewline
Ciflorette & 1 & 3 & 0.8956203 & 6 & 3.076745 & 0 & 0.0000000 &
115\tabularnewline
Ciflorette & 2 & 4 & 1.0108469 & 3 & 3.215120 & 0 & 0.0000000 &
78\tabularnewline
Ciflorette & 3 & 3 & 0.6290460 & 5 & 2.495157 & 1 & 0.8462441 &
31\tabularnewline
Ciflorette & 4 & 4 & 1.5434873 & 5 & 3.041381 & 2 & 0.7812132 &
17\tabularnewline
Ciflorette & 5 & 5 & 2.3094011 & 3 & 5.196152 & 1 & 0.5773503 &
3\tabularnewline
Clery & 0 & 8 & 2.9652070 & 10 & 6.909543 & 2 & 1.2462382 &
54\tabularnewline
Clery & 1 & 3 & 1.0363172 & 4 & 3.430427 & 0 & 0.1010153 &
98\tabularnewline
Clery & 2 & 3 & 0.7792759 & 2 & 2.378862 & 0 & 0.3461440 &
63\tabularnewline
Clery & 3 & 3 & 0.6485965 & 3 & 2.321718 & 0 & 0.6969503 &
34\tabularnewline
Clery & 4 & 3 & 0.7703289 & 2 & 2.139375 & 1 & 0.6992932 &
14\tabularnewline
Capriss & 0 & 10 & 1.8239229 & 9 & 5.912624 & 2 & 0.9705661 &
54\tabularnewline
Capriss & 1 & 3 & 1.0404784 & 3 & 2.321914 & 0 & 0.0000000 &
190\tabularnewline
Capriss & 2 & 4 & 0.9443864 & 1 & 1.983821 & 0 & 0.0990148 &
102\tabularnewline
Capriss & 3 & 3 & 0.7063460 & 2 & 1.709556 & 0 & 0.4016097 &
31\tabularnewline
Capriss & 4 & 2 & 1.0000000 & 1 & 1.154700 & 0 & 0.5000000 &
4\tabularnewline
Darselect & 0 & 6 & 2.2875105 & 7 & 6.271880 & 1 & 1.1060156 &
54\tabularnewline
Darselect & 1 & 3 & 1.0764669 & 5 & 4.455098 & 0 & 0.1072113 &
87\tabularnewline
Darselect & 2 & 3 & 0.9597149 & 4 & 2.263362 & 0 & 0.3422980 &
57\tabularnewline
Darselect & 3 & 3 & 0.6803587 & 3 & 2.593462 & 0 & 0.4845800 &
39\tabularnewline
Darselect & 4 & 3 & 2.0701967 & 3 & 2.121320 & 1 & 1.0606602 &
8\tabularnewline
Darselect & 5 & 2 & NA & 5 & NA & 1 & NA & 1\tabularnewline
Cir107 & 0 & 10 & 3.2868882 & 10 & 8.522261 & 2 & 1.3282134 &
54\tabularnewline
Cir107 & 1 & 4 & 1.6110796 & 6 & 5.063588 & 0 & 0.2543235 &
154\tabularnewline
Cir107 & 2 & 3 & 0.9382965 & 3 & 3.698939 & 0 & 0.0000000 &
110\tabularnewline
Cir107 & 3 & 3 & 0.7079686 & 4 & 2.880760 & 0 & 0.5587442 &
41\tabularnewline
Cir107 & 4 & 4 & 2.1001701 & 4 & 2.531939 & 1 & 0.7559289 &
8\tabularnewline
\bottomrule
\end{longtable}

\subsubsection{Number of Module for successive
orders}\label{number-of-module-for-successive-orders}

\begin{Shaded}
\begin{Highlighting}[]
\NormalTok{tab1<-}\StringTok{ }\KeywordTok{fc_dist_module_by_order}\NormalTok{(}\DataTypeTok{data =}\NormalTok{ dat)}
\KeywordTok{kable}\NormalTok{(}\DataTypeTok{x =}\NormalTok{ tab1,}
      \DataTypeTok{caption =} \StringTok{"No. Module by varieties for successive orders "}
\NormalTok{      )}
\end{Highlighting}
\end{Shaded}

\begin{longtable}[]{@{}lrrrrrrr@{}}
\caption{No. Module by varieties for successive orders}\tabularnewline
\toprule
& 0 & 1 & 2 & 3 & 4 & 5 & Frequency\tabularnewline
\midrule
\endfirsthead
\toprule
& 0 & 1 & 2 & 3 & 4 & 5 & Frequency\tabularnewline
\midrule
\endhead
Gariguette & 54 & 94 & 62 & 21 & 9 & 1 & 241\tabularnewline
Ciflorette & 54 & 115 & 78 & 31 & 17 & 3 & 298\tabularnewline
Clery & 54 & 98 & 63 & 34 & 14 & 0 & 263\tabularnewline
Capriss & 54 & 190 & 102 & 31 & 4 & 0 & 381\tabularnewline
Darselect & 54 & 87 & 57 & 39 & 8 & 1 & 246\tabularnewline
Cir107 & 54 & 154 & 110 & 41 & 8 & 0 & 367\tabularnewline
Frequency & 324 & 738 & 472 & 197 & 60 & 5 & 1796\tabularnewline
\bottomrule
\end{longtable}

\subsubsection{Occurence of the higher order along
time}\label{occurence-of-the-higher-order-along-time}

\begin{Shaded}
\begin{Highlighting}[]
\NormalTok{tab2<-}\StringTok{ }\KeywordTok{fc_dist_order_by_date}\NormalTok{(}\DataTypeTok{data =}\NormalTok{ dat,}
                             \DataTypeTok{genotype =} \StringTok{"Gariguette"}\NormalTok{,}
                             \DataTypeTok{prob =}\NormalTok{ F)}
\KeywordTok{kable}\NormalTok{(}\DataTypeTok{x =}\NormalTok{ tab2, }
      \DataTypeTok{caption =} \StringTok{"Module order frequency distribution for successive date"}\NormalTok{,}\DataTypeTok{digits =} \DecValTok{2}\NormalTok{)}
\end{Highlighting}
\end{Shaded}

\begin{longtable}[]{@{}lrrrrrrr@{}}
\caption{Module order frequency distribution for successive
date}\tabularnewline
\toprule
& Mid-December & Early-Junuary & Mid-February & Early-March &
Early-April & Early-June & Frequency\tabularnewline
\midrule
\endfirsthead
\toprule
& Mid-December & Early-Junuary & Mid-February & Early-March &
Early-April & Early-June & Frequency\tabularnewline
\midrule
\endhead
0 & 9 & 9 & 9 & 9 & 9 & 9 & 54\tabularnewline
1 & 0 & 7 & 24 & 17 & 26 & 20 & 94\tabularnewline
2 & 0 & 0 & 0 & 11 & 28 & 23 & 62\tabularnewline
3 & 0 & 0 & 0 & 1 & 3 & 17 & 21\tabularnewline
4 & 0 & 0 & 0 & 0 & 0 & 9 & 9\tabularnewline
5 & 0 & 0 & 0 & 0 & 0 & 1 & 1\tabularnewline
Frequency & 9 & 16 & 33 & 38 & 66 & 79 & 241\tabularnewline
\bottomrule
\end{longtable}


\end{document}
